\documentclass{article}

\usepackage{vaucanson-g}
\usepackage{tikz}
\usetikzlibrary{matrix}

\newcommand{\nodelabel}[2]{\mbox{$#1$-$#2$}}
\newcommand{\varName}[1]{\textrm{\it#1}}

\begin{document}

\begin{VCPicture}{(-1,-12)(10,0)}

% Set up so that each state needs an explicit line width, line width is only reset at the end of state definition block  
\ChgStateLineWidth{1.5} \FinalStateVar[\nodelabel{A}{1}]{(4,0)}{A0}
\ChgStateLineWidth{1.5} \FinalStateVar[\nodelabel{B}{2}]{(2,-2)}{B1} \ChgStateLineWidth{0.3} \StateVar[\nodelabel{C}{3}]{(6,-2)}{C1}
\ChgStateLineWidth{0.3} \StateVar[\nodelabel{A}{4}]{(0,-4)}{A2} \ChgStateLineWidth{1.5} \FinalStateVar[\nodelabel{C}{5}]{(4,-4)}{C2} 
\ChgStateLineWidth{0.3} \StateVar[\nodelabel{A}{6}]{(2,-6)}{A3} \ChgStateLineWidth{1.5} \FinalStateVar[\nodelabel{B}{7}]{(6,-6)}{B3}
\ChgStateLineWidth{1.5} \FinalStateVar[\nodelabel{A}{8}]{(4,-8)}{A4} \ChgStateLineWidth{0.3} \StateVar[\nodelabel{C}{9}]{(8,-8)}{C4}
\ChgStateLineWidth{0.3} \StateVar[\nodelabel{B}{10}]{(2,-10)}{B5} \ChgStateLineWidth{0.3} \StateVar[\nodelabel{C}{11}]{(6,-10)}{C5}
\RstStateLineWidth

% Set up so that after each line of tex code all parameters are at their default setting
\EdgeR{A0}{B1}{} \ChgEdgeLineStyle{dashed} \EdgeL{A0}{C1}{} \RstEdgeLineStyle
\ChgEdgeLineStyle{dashed} \EdgeR{B1}{A2}{} \RstEdgeLineStyle \EdgeL{B1}{C2}{}
\ChgEdgeLineStyle{dashed} \EdgeR{C2}{A3}{} \RstEdgeLineStyle \EdgeL{C2}{B3}{}
\EdgeR{B3}{A4}{}  \ChgEdgeLineStyle{dashed} \EdgeL{B3}{C4}{} \RstEdgeLineStyle
\ChgEdgeLineStyle{dashed} \EdgeR{A4}{B5}{} \EdgeL{A4}{C5}{} \RstEdgeLineStyle

\end{VCPicture}

\iffalse

\begin{tikzpicture}[scale = 0.75, every node/.style = {scale = 0.75}]
\matrix (m) [matrix of nodes,
nodes in empty cells,
column sep = 0, row sep = 0,
column 2/.style = {nodes = {text width = 3.5cm, align = left, minimum width = 3.5cm}}]
{
	\textit{A} & \nodelabel{@}{1} \ \nodelabel{B}{2} \ \nodelabel{C}{5} \ \nodelabel{@}{8} \\
	\textit{B} & \nodelabel{@}{2} \ \nodelabel{@}{7} \ \nodelabel{A}{8} \\
	\textit{C} & \nodelabel{A}{1} \ \nodelabel{@}{5} \ \nodelabel{B}{7} \ \nodelabel{A}{8} \\
};

\draw (m-1-1.north west) rectangle (m-3-1.south east);
\draw (m-1-1.north west) rectangle (m-3-2.south east);

\foreach \x in {1,2} {
	\draw (m-\x-1.south west)  --  (m-\x-2.south east);
};
\end{tikzpicture}

Here is an example to illustrate the concept of our breadth-first history and how it lends itself to the rate evaluation for the tree. We build the breadth-first history as follows: We iterate through the tree in a breadth-first order. For each node in the supertree, we update the node type's linked list in the breadth-first history. Whether the node is failed or not determines what insertion we make at the end of the linked list. If we encounter a node that has failed, we add the symbol @, whereas if we encounter a node that has not failed, we add the parent of that node to the end of the list. The tree contains only double circled nodes, and the supertree includes all nodes. Each node label contains the type of the component and a breadth-first order ID. The breadth-first history obtained for the supertree is given below it. The breadth-first order IDs have been given for convenience in seeing the correspondence between the supertree and its breadth-first history.

Consider a model with $compSet = \{A, B, C\}$, where redundancy of each component type is 4. Each component type can cause all other component types to fail which gives us the following $\Gamma$ definitions: $\Gamma_{A} = \{B, C\}$, $\Gamma_{B} = \{A, C\}$ and $\Gamma_{C} = \{A, B\}$. With this model, consider a failure transition with ``from" state: (2, 2, 2, $\mathcal{E}$) and ``to" state: (4, 4, 3, $\mathcal{E}$), where the first three numbers represent the number of failed components of type $A$, $B$, and $C$ respectively, and $\mathcal{E}$ is the environment. The tree and supertree shown here correspond to the given failure transition. Rate calculation for the given tree proceeds as follows:

compsAvailable for a type, $\zeta_{Type}$, is initialized with Redundancy(Type) -  $x(\mbox{Type})$, where $x$ is the ``from" state and Type is the component type. In our case, $n = 2$. Failure rate of the root is the product of  $\zeta_{RootType}$, $n$, and $\lambda_{A, \mathcal{E}}$.

For evaluating the failure rate of the tree, we take the product of $\phi_{i,j}$ for all edges from a parent \varName{i} to a child \varName{j} in the tree. Hence, the failure rate of the tree equals $\phi_{A, B}\, \phi_{B, C}\, \phi_{C, B}\, \phi_{B, A}$.

Now we calculate the cumulative complement probability of nodes, or the supertree rate, that did not fail from breadth-first history.
We iterate through the linked list for each type in the breadth-first history as follows:
For each @ we reduce  $\zeta_{type}$ for the component type by 1, where Type is used to index the breadth-first history as well. As long as  $\zeta_{type} > 0$ we include $1 - \phi_{parent, type}$ term in the supertree rate. We have  $\zeta_{B} = 2$ and  $\zeta_{C}= 2$.
Iterating through at index $A$: $\zeta_{A} = 1, (1 - \phi_{B, A}) (1 - \phi_{C, A}), \zeta_{A} = 0$
Iterating through at index $B$: $\zeta_{B} = 1, \zeta_{B} = 0$
Iterating through at index $C$: $ (1 - \phi_{A, C}), \zeta_{C} = 1, (1 - \phi_{B, C}) (1 - \phi_{A, C}) $
Multiplying all the individual rates, final rate for tree, given the transition is $ n \lambda_{A, 1} \phi_{A, B} \phi_{B, C} \phi_{C, B} * \phi_{B, A} * (1 - \phi_{B, A}) (1 - \phi_{C, A}) (1 - \phi_{A, C}) (1 - \phi_{B, C}) (1 - \phi_{A, C}) $.

This calculated rate then is added to the entry in the $Q$-matrix that corresponds to the failure transition given.

\fi

\end{document}