\documentclass{article}

\usepackage{vaucanson-g}
\usepackage{tikz}
\usetikzlibrary{matrix}

\newcommand{\nodelabel}[2]{\mbox{$#1$-$#2$}}

\begin{document}

\begin{VCPicture}{(-1,-12)(10,0)}

% Set up so that each state needs an explicit line width, line width is only reset at the end of state definition block  
\ChgStateLineWidth{1.5} \FinalStateVar[\nodelabel{A}{1}]{(4,0)}{A0}
\ChgStateLineWidth{1.5} \FinalStateVar[\nodelabel{B}{2}]{(2,-2)}{B1} \ChgStateLineWidth{0.3} \StateVar[\nodelabel{C}{3}]{(6,-2)}{C1}
\ChgStateLineWidth{0.3} \StateVar[\nodelabel{A}{4}]{(0,-4)}{A2} \ChgStateLineWidth{1.5} \FinalStateVar[\nodelabel{C}{5}]{(4,-4)}{C2} 
\ChgStateLineWidth{0.3} \StateVar[\nodelabel{A}{6}]{(2,-6)}{A3} \ChgStateLineWidth{1.5} \FinalStateVar[\nodelabel{B}{7}]{(6,-6)}{B3}
\ChgStateLineWidth{1.5} \FinalStateVar[\nodelabel{A}{8}]{(4,-8)}{A4} \ChgStateLineWidth{0.3} \StateVar[\nodelabel{C}{9}]{(8,-8)}{C4}
\ChgStateLineWidth{0.3} \StateVar[\nodelabel{B}{10}]{(2,-10)}{B5} \ChgStateLineWidth{0.3} \StateVar[\nodelabel{C}{11}]{(6,-10)}{C5}
\RstStateLineWidth

% Set up so that after each line of tex code all parameters are at their default setting
\EdgeR{A0}{B1}{} \ChgEdgeLineStyle{dashed} \EdgeL{A0}{C1}{} \RstEdgeLineStyle
\ChgEdgeLineStyle{dashed} \EdgeR{B1}{A2}{} \RstEdgeLineStyle \EdgeL{B1}{C2}{}
\ChgEdgeLineStyle{dashed} \EdgeR{C2}{A3}{} \RstEdgeLineStyle \EdgeL{C2}{B3}{}
\EdgeR{B3}{A4}{}  \ChgEdgeLineStyle{dashed} \EdgeL{B3}{C4}{} \RstEdgeLineStyle
\ChgEdgeLineStyle{dashed} \EdgeR{A4}{B5}{} \EdgeL{A4}{C5}{} \RstEdgeLineStyle

\end{VCPicture}

\begin{tikzpicture}[scale = 2.5, every node/.style = {scale = 2.5}]

\matrix (m) [matrix of nodes,
nodes in empty cells,
column sep = 0, row sep = 0,
column 2/.style = {nodes = {text width = 3.5cm, align = left, minimum width = 3.5cm}}]
{
	\textit{A} & \nodelabel{@}{1} \ \nodelabel{B}{2} \ \nodelabel{C}{5} \ \nodelabel{@}{8} \\
	\textit{B} & \nodelabel{@}{2} \ \nodelabel{@}{7} \ \nodelabel{A}{8} \\
	\textit{C} & \nodelabel{A}{1} \ \nodelabel{@}{5} \ \nodelabel{B}{7} \ \nodelabel{A}{8} \\
};

\draw (m-1-1.north west) rectangle (m-3-1.south east);
\draw (m-1-1.north west) rectangle (m-3-2.south east);

\foreach \x in {1,2} {
	\draw (m-\x-1.south west)  --  (m-\x-2.south east);
};

\end{tikzpicture}
 
Here is an example to illustrate the concept of our Breadth First History and how it lends itself to the rate evaluation for the subTree. We build the Breadth First History as follows: We iterate through the tree in a Breadth First Order. Whether the node is failed or not we always update the linked list at the index of this node's type in the Breadth First History. Whether the node is failed or not determines what we add at the end of the linked list. If we encounter a node that has failed we add @, whereas if we encounter a node that is not failed we add the parent of that node to the end of the list. The Breadth First History obtained for the tree is given below it. The nodes in the tree have been given IDs for convenience in seeing the correspondence between the tree and its Breadth First History.

The tree shown here is one of the trees that corresponds to this failure transition. Rate calculation proceeds as follows. \\
% symbolic math for tree
Given: \\ 
$Component Set = \{A, B, C\}$ \\
Redundancy(A) = 4, Redundancy(B) = 4, Redundancy(C) = 4 \\
$\Gamma_{A} = \{B, C\}$ \\
$\Gamma_{B} = \{A, C\}$ \\
$\Gamma_{C} = \{A, B\}$ \\
from state (2, 2, 2) \\ 
to state (4, 4, 3) \\
$\varepsilon = \{1\}$ \\
Rate calculation for the root: \\ $n = 2$ given by subtracting count of A (root component) in from redundancy of A. \\ 
Failure rate of root $= n * \lambda_{A, 1}$ \\
Failure rate of subTree $= \phi_{A, B} * \phi_{B, C} * \phi_{C, B} * \phi_{B, A} $ (deals with failed nodes only)\\
Failure rate from Breadth First History will be calculated as follows. \\
Available(Type) = Redundancy(Type) - From(Type)
This gives \\ Available(A) = 2, Available(B) = 2, Available(C) = 2 \\
We iterate through the linked list for each type in the Breadth First History as follows: 
For each @ we reduce Available(Type) by 1. As long as Available(Type) $> 0$ we include $1 - \phi_{parent, type}$. \\
Iterating through at index A: Available(A) = 1, $ (1 - \phi_{B, A}) * (1 - \phi_{C, A}) $, Available(A) = 0 \\
Iterating through at index B: Available(B) = 1, Available(B) = 0 (no contribution) \\
Iterating through at index C: $ (1 - \phi_{A, C}) $, Available(C) = 1, $ (1 - \phi_{B, C}) * (1 - \phi_{A, C}) $ \\
Multiplying all the individual rates, final rate for tree, given the transition is $ n * \lambda_{A, 1} * \phi_{A, B} * \phi_{B, C} * \phi_{C, B} * \phi_{B, A} * (1 - \phi_{B, A}) * (1 - \phi_{C, A}) * (1 - \phi_{A, C}) $ * $ (1 - \phi_{B, C}) * (1 - \phi_{A, C}) $. 
\end{document}