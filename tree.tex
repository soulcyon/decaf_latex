\documentclass{article}

\usepackage{vaucanson-g}
\usepackage{tikz}
\usetikzlibrary{matrix}

\newcommand{\nodelabel}[2]{\mbox{$#1$-$#2$}}

\begin{document}

\begin{VCPicture}{(-1,-12)(10,0)}

% Set up so that each state needs an explicit line width, line width is only reset at the end of state definition block  
\ChgStateLineWidth{1.5} \FinalStateVar[\nodelabel{A}{1}]{(4,0)}{A0}
\ChgStateLineWidth{1.5} \FinalStateVar[\nodelabel{B}{2}]{(2,-2)}{B1} \ChgStateLineWidth{0.3} \StateVar[\nodelabel{C}{3}]{(6,-2)}{C1}
\ChgStateLineWidth{0.3} \StateVar[\nodelabel{A}{4}]{(0,-4)}{A2} \ChgStateLineWidth{1.5} \FinalStateVar[\nodelabel{C}{5}]{(4,-4)}{C2} 
\ChgStateLineWidth{0.3} \StateVar[\nodelabel{A}{6}]{(2,-6)}{A3} \ChgStateLineWidth{1.5} \FinalStateVar[\nodelabel{B}{7}]{(6,-6)}{B3}
\ChgStateLineWidth{1.5} \FinalStateVar[\nodelabel{A}{8}]{(4,-8)}{A4} \ChgStateLineWidth{0.3} \StateVar[\nodelabel{C}{9}]{(8,-8)}{C4}
\ChgStateLineWidth{0.3} \StateVar[\nodelabel{B}{10}]{(2,-10)}{B5} \ChgStateLineWidth{0.3} \StateVar[\nodelabel{C}{11}]{(6,-10)}{C5}
\RstStateLineWidth

% Set up so that after each line of tex code all parameters are at their default setting
\EdgeR{A0}{B1}{} \ChgEdgeLineStyle{dashed} \EdgeL{A0}{C1}{} \RstEdgeLineStyle
\ChgEdgeLineStyle{dashed} \EdgeR{B1}{A2}{} \RstEdgeLineStyle \EdgeL{B1}{C2}{}
\ChgEdgeLineStyle{dashed} \EdgeR{C2}{A3}{} \RstEdgeLineStyle \EdgeL{C2}{B3}{}
\EdgeR{B3}{A4}{}  \ChgEdgeLineStyle{dashed} \EdgeL{B3}{C4}{} \RstEdgeLineStyle
\ChgEdgeLineStyle{dashed} \EdgeR{A4}{B5}{} \EdgeL{A4}{C5}{} \RstEdgeLineStyle

\end{VCPicture}

\begin{tikzpicture}[scale = 2.5, every node/.style = {scale = 2.5}]

\matrix (m) [matrix of nodes,
nodes in empty cells,
column sep = 0, row sep = 0,
column 2/.style = {nodes = {text width = 3.5cm, align = left, minimum width = 3.5cm}}]
{
	\textit{A} & \nodelabel{@}{1} \ \nodelabel{B}{2} \ \nodelabel{C}{5} \ \nodelabel{@}{8} \\
	\textit{B} & \nodelabel{@}{2} \ \nodelabel{@}{7} \ \nodelabel{A}{8} \\
	\textit{C} & \nodelabel{A}{1} \ \nodelabel{@}{5} \ \nodelabel{B}{7} \ \nodelabel{A}{8} \\
};

\draw (m-1-1.north west) rectangle (m-3-1.south east);
\draw (m-1-1.north west) rectangle (m-3-2.south east);

\foreach \x in {1,2} {
	\draw (m-\x-1.south west)  --  (m-\x-2.south east);
};

\end{tikzpicture}

\end{document}