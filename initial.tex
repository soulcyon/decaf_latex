\documentclass[12pt]{article}
\author{Mihir Sanghavi}
\date{November 2012}

\usepackage{algorithm}
\usepackage{algorithmic}
\usepackage{mathabx}

\newcommand{\VARNAME}[1]{\textrm{\it#1}}

\begin{document}

\begin{algorithm}
\caption{SeedSubTrees($\Gamma$)}
\label{alg:seedsubtrees}
\begin{algorithmic}[1]
\FOR{\VARNAME{comp} $\in$ \VARNAME{compSet}}
  \IF{(Empty($\Gamma_{\VARNAME{comp}}$))}
    \STATE continue;
  \ENDIF
  \STATE \VARNAME{level} = [ ]; \COMMENT {dynamic array of failed components at subTree level}
  \STATE \VARNAME{nFailed} = $(0, 0, \ldots, 0)$; \COMMENT {counts failed components}
  \STATE \VARNAME{BFHist} = $ (\{\}, \{\}, \ldots, \{\}) $; \COMMENT {an array of linked lists that keeps a breadth first history of subTrees, array is indexed by component, linked list for each component stores parents in breadth first order}
  %\STATE $\Gamma Cache = \{$Array of Size $2 ^{|\Gamma_{comp}|}\};$ \COMMENT {Stores All Possible Subsets of $\Gamma_{comp}$} 
  \STATE add \VARNAME{comp} to \VARNAME{level};
  \STATE \VARNAME{nFailed}[\VARNAME{comp}] = 1;
  %\STATE $\Gamma Cache_{comp} = \mathcal{P}(\Gamma_{comp});$ \COMMENT{Power Set (Ordered Set) is Cached}
  \STATE add \# to  \VARNAME{BFHist}[\VARNAME{comp}]; \COMMENT {signifies one component of type comp has failed}
  \STATE AddSubTreeLevel(\VARNAME{level}, \VARNAME{nFailed}, \VARNAME{BFHist}, 1, \VARNAME{comp});
\ENDFOR
\end{algorithmic}
\end{algorithm}

\begin{algorithm}
\caption{AddSubTreeLevel(\VARNAME{level}, \VARNAME{nFailed}, \VARNAME{BFHist}, \VARNAME{subTreeRate}, \VARNAME{rootC})}
\label{alg:growsubtree}
\begin{algorithmic}[1]
\STATE \VARNAME{nextLevelPossibilities} = $\bigtimes\limits_{i\,=\,1}^{|level|}\mathcal{P}(\Gamma_{level[i]});$ 
\\ \COMMENT{Cartesian Product (Power Sets (Ordered Set))}
  \FOR{\VARNAME{oneNextLevelPossibility} $\in$ \VARNAME{nextLevelPossibilities}}
    \STATE \VARNAME{childAddedFlag} = False;
    \FOR{\VARNAME{parentC} $\in$ \VARNAME{level}}
      \FOR{\VARNAME{childC} $\in$ $\Gamma_{\VARNAME{parentC}}$}
          \IF{\VARNAME{childC} $\in$ \VARNAME{oneNextLevelPossibility}}  
            \IF{\VARNAME{nFailed}[\VARNAME{childC}] == Redundancy(\VARNAME{childC})}
              \STATE goto line 3; 
              \COMMENT {invalid subtree, requires more comps than available in system}
            \ENDIF 
            \STATE \VARNAME{childAddedFlag} = True;
            \STATE \VARNAME{nFailed}[\VARNAME{childC}] = \VARNAME{nFailed}[\VARNAME{childC}] + 1;
            \STATE add \# to \VARNAME{BFHist}[\VARNAME{comp}]; \COMMENT {signifies one component of type comp has failed}
            \STATE $\VARNAME{subTreeRate} = \VARNAME{subTreeRate} * \phi_{parentC,\; childC};$ 
            \\ \COMMENT {update rate with $\phi$}
          \ELSE
           \STATE add \VARNAME{parentC} to \VARNAME{BFHist}[\VARNAME{childC}]; \COMMENT {signifies one component of type comp has not failed}
         \ENDIF
      \ENDFOR
    \ENDFOR
    \STATE
    \IF{\VARNAME{childAddedFlag}}
      \STATE AddSubTreeLevel(\VARNAME{oneNextLevelPossibility}, \VARNAME{nFailed}, \VARNAME{BFHist}, \VARNAME{subTreeRate}, \VARNAME{rootC}); 
      \\ \COMMENT {subTree can be grown further}
    \ELSE
      \STATE ProcessRates(\VARNAME{nFailed}, \VARNAME{BFHist}, \VARNAME{subTreeRate}, \VARNAME{rootC});
      \\ \COMMENT {current subTree is completed because it cannot be grown further}
    \ENDIF
\ENDFOR
\end{algorithmic}
\end{algorithm}

\begin{algorithm}
\caption{ProcessRates(\VARNAME{nFailed}, \VARNAME{BFHist}, \VARNAME{subTreeRate}, \VARNAME{rootC})}
\label{alg:processrates}
\begin{algorithmic}[1]
\FOR{$\VARNAME{x} \in \VARNAME{Q}$}
  \STATE \VARNAME{y} = \VARNAME{x};
  \STATE \VARNAME{e} = Environment(x);
  \FOR{\VARNAME{comp} $\in$ \VARNAME{compSet}}
  \STATE \VARNAME{y}[\VARNAME{comp}] = \VARNAME{y}[\VARNAME{comp}] + \VARNAME{nFailed}[\VARNAME{comp}];
  \ENDFOR
  % \STATE Environment$(\VARNAME{y}) = \:$Environment$(\VARNAME{x});$
  \IF{\VARNAME{y} is not a valid state}
  \STATE continue;
  \ENDIF
  % \STATE $\VARNAME{n} = \:$Redundancy$(\VARNAME{rootC}) - \VARNAME{x}[\VARNAME{rootC}];$
  % \STATE $\VARNAME{rootFailureRate} = \VARNAME{n} * \lambda_{\VARNAME{rootC},\; \VARNAME{env}}\,;$
  \STATE $\VARNAME{rootFailureRate} = \:($Redundancy$(\VARNAME{rootC}) - \VARNAME{x}[\VARNAME{rootC}]) * \lambda_{\VARNAME{rootC},\; \VARNAME{e}}\,;$
  \STATE $\VARNAME{cumNotFailedProb} = 1;$ \COMMENT{cumulative probability of comps that could have failed but did not}
  \FOR{$\VARNAME{comp} \in \{\VARNAME{Components}\}$}
    \STATE $\VARNAME{compsAvailable} = \:$Redundancy$(\VARNAME{comp}) - \VARNAME{x}[\VARNAME{comp}];$
    \FOR{$\VARNAME{parentC} \in \VARNAME{BFHist}[\VARNAME{comp}]$}
      \IF{\VARNAME{parentC} == $|$}
        \STATE $\VARNAME{compsAvailable} = \VARNAME{compsAvailable} - 1;$
      \ELSIF {$\VARNAME{compsAvailable} > 0$} 
        \STATE $\VARNAME{cumNotFailedProb} = \VARNAME{cumNotFailedProb} * (1 - \phi_{parentC,\; comp});$
      \ENDIF
    \ENDFOR
  \ENDFOR
  \STATE $\VARNAME{Q}(\VARNAME{x},\VARNAME{y}) = \VARNAME{rootFailureRate} * \VARNAME{subTreeRate} * \VARNAME{complementRate};$
\ENDFOR
\end{algorithmic}
\end{algorithm}
\end{document}